\section{Security}
Concerns related to security were, as they typically are in this day and age, a thing.
Several issues arise as the platform allows users to control agents by issuing commands through a publicly available API, namely:

% Agent = active/connected robot/entity/thing
% user = bruker ja, ok?
% user client = bruker som tar i bruk webgrensesnittet
% server = serveren
% client = både agenter og user clients
\begin{enumerate}
	\item Restricting execution of commands.
	\item Preventing replay of commands (replay attack).
	\item Preventing others from masquerading as the server and spreading misinformation (man-in-the-middle type attack).
	\item And that's it. I can't think of any other issues.
\end{enumerate}
% I AM A PINK PONY
% A BIG FAT PHONY
% SEE THE CAVEAT OF MY CALZONE
% FEMALES BE TRIPPIN AND I'M LIKE YEAH THAT'S ON ME

% jobbe deg oppover mot sluttresultatet

\subsection{Communication}
Communication between the server and clients is happens over HTTP.
HTTP is like, a protocol for transfering text.
\emph{Hyper}-text.
This is well, done un-encrypted.
So we use HTTPS instead which is like HTTP except with end-to-end encryption by virtue of being layered on top of SSL/TLS.
This delegates the burden of securing our system's communication to someone else, and should prevent potential attackers from eavesdropping on our system.
%TODO: find reference for SSL/TLS Certificate thing type deal
%TODO: find a paper or whatever (RFC?) on HTTPS and SSL/TLS

\subsection{Identity}
The use of HTTPS requires a digital certificate. %TODO: REF to something about certificates 
This makes it difficult for potential attackers to masquerade as the service, even if they somehow manage to take control of its domain.
An attacker might want to do so in order to feed the users' faulty sensor readings or to capture user credentials.

\subsection{Access restriction}
%TODO: Access Control Lists, mention theoretical implementation

\subsection{User identification}
Users register for the service and leave a public key on the server along with a username and password.
The username and password are used to authenticate to gain access to the web interface and to make changes to one's account and registered agents.

When users send a command to the server they sign the command using their private key and attach the signature to the message along with their user ID.
The signature is verified using the public key belonging to the given user ID when the message arrives at the server.
This allows the server to verify that the command was issued by the given user, and that said user is allowed to execute the commands.
Because of HTTPS, it is non-trivial to intercept the commands. 

%TODO: write about digital signatures and stuff?

%TODO: make a graph or overview of the message structure. Should probably put it somewhere else.

