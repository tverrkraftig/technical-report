\section{Robot}
To demonstrate the platform it was decided to make a vehicle with a manipulator controlled remotely over internet. In addition the robot should have sensors that send data (if available) continuously to the server and be able to send camera feed to the operator. Due to the possibility to expand this later with more complex functionality, the program was written in C++.
The following was used to make this vehicle:
\begin{itemize}
    \item Raspberry Pi
    \item Raspberry Pi Camera
    \item Dynamixel AX-12 Servomotors
    \item Dynamixel AX-S1 Integrated Sensor
    \item USB2Dynamixel
    \item Dynamixel SDK
\end{itemize}
\vspace{\secspace}

\textbf{Rasppberry Pi} (Pi) is a single board credit-card-sized computer. It runs on an 700 MHz ARM processor.
It has two USB inputs, ethernet, HDMI and gpio (general purpose input output) headers. 
This makes it the perfect prototyping computer for this kind of project. 
The Pi is running its operating system from a SD card, and the OS is RaspBian which is a debian based linux distro.

\begin{figure}[H]
    \centering
    \includegraphics[width=0.7\textwidth]{graphics/Raspberry_Pi.png}	
    \caption{Raspberry Pi overview over peripherals}
    \label{fig:RPi}
\end{figure}

To control the Pi you can hook up a keyboard and a monitor directly to the board, or you can control it from another computer. 
This is achieved by using a SSH server on the Pi and a SSH client on the computer. 
SSH (Secure SHell) is a protocol that allows one computer to remotely controll another via command line. A widely known program for doing this is PuTTY. To login over SSH, you need to know the IP-address of the host. This can be done by setting the IP-address static or to scan the network to find out which IP the Pi would have. Since both of these approaches are difficult on a big network like NTNU, we had to find another approach. The solution was to connect the Pi directly to the computer via an ethernet cable. To do this we had to do some modifications, the full tutorial on how to do this see \footnote{\url{http://pihw.wordpress.com/guides/direct-network-connection/}}. To get internet access we first shared the Wi-Fi connection on the computer described here \footnote{\url{http://anwaarullah.wordpress.com/2013/08/12/sharing-wifi-internet-connection-with-raspberry-pi-through-lanethernet-headless-mode/}}. Later we used a Wi-Fi usb dongle. The driver for this dongle installed automatically and all we had to do was to configure \textit{/etc/network/interfaces}, and add the following:
\begin{verbatim}
    auto wlan0
    iface wlan0 inet dhcp
    wpa-ssid "<nameOfNetwork>"
    wpa-psk "<networkPassword>"
\end{verbatim}
We had to set up a own network to do this, because the eduroam network uses different setup.
\vspace{\secspace}

\textbf{Raspberry Pi camera}

\vspace{\secspace}

\textbf{Dynamixel AX-12 servomotors}
are motors that allows for precise controll of angle and velocity. These motors are controlled over a half duplex UART, which is a byte oriented asynchronous serial communication protocol. You can control the motors and receive feedback by sending commands to it corresponding to the control table in the datasheet [PUT REFZ H3R3]. The most important features is the control of position and velocity. The servos can work like normal servos where you can put in the desired position, and a controller inside the servo will make the servo go to that position. This position is limited to between 0-300 degrees (see datasheet). The servo can also work in so called "Endless Turn" mode, where the servos can spin infinite. Here you can only control the velocity which the motors run. "Endless Turn" mode is activated by setting the angle limits (CW Angle Limit and CCW Angle Limit) to zero.
\vspace{\secspace}

\textbf{Dynamixel AX-S1 Integrated Sensor}
is a sensor device capable of measuring sound, brightness, heat and distance to objects. It is also capable of making sound. The communication is the same as the servomotors and the sensor is connected to the same bus. 
\vspace{\secspace}

\textbf{USB2Dynamixel} \footnote{\url{http://support.robotis.com/en/product/auxdevice/interface/usb2dxl_manual.htm}} is a USB device that allows the computer to create a virtual serial port (UART) and with some other circuitry, communicate with the servos over the USB port.
This USB requires no driver installation when running linux, and since RaspBian is a linux distro this simplifies things. 
The USB2Dynamixel is inserted into one of the USB ports on the Pi and the servor motors are connected to the device.

Since the Pi also has a hardware UART driver on two of its gpio headers, we thought about if we could communicate with the servos through these pins. 
To do this we had to make the circuitry described on page 8 in the datasheet [PUT REFZ H3R3]. 
We would also have to implement our own code for the lower part of the communication (where we used a library from the manufacturer, more on that later).
UART is also widely supported by many lower end microcontrollers which don't have the support for an OS. 
Therefore using UART would mean that the code would be even more platform independent.
\vspace{\secspace}

\textbf{Dynamixel SDK} is a programming library for controlling dynamixel servo motors. 
This library is available for Windows, Mac and Linux and easy to run on the Pi. 
Since the library is written in C it is easy to use it in this C++ project.
There is a great API reference \footnote{\url{http://support.robotis.com/en/software/dynamixelsdk.htm}} with the library.
The functions treat the lower part of the communication over USB2Dynamixel.
There are functions for initializing the communication, terminate the communication, sending byte and words (16 bit), receiving byte and words, and for ping. Ping is for checking if a device is connected.
There is also a page describing platform porting \footnote{\url{http://support.robotis.com/en/software/dynamixel_sdk/sourcestructure.htm}}


\subsection{The overall system}

\begin{figure}[H]
    \centering
    \includegraphics[width=0.7\textwidth]{graphics/Robot_class_diagram.png}	
    \caption{Robot overall system}
    \label{fig:Class_diagram}
\end{figure}

This is how the overall system was made. It was our intention to make the system module based, such that it is easy to develop more things using the same modules. To make the modules we implemented classes in c++. That way you can make several motor, sensor, car or manipulator objects. 

The communication module was added because we used threads in our program. Since there are only one data-bus and therefore it can only be accessed by one thread, we had to implement \textit{Mutexes} such that only one thread can access the communication functions available in the dynamixel library at the time. \textit{Mutexes} are "locks" that protect such code blocks that can only be accessed by one at the time.

\subsection{Wiggle It}

The first step is to make a servo motor move. 

\begin{figure}[H]
    \centering
    %sett inn figure her
    \caption{Servo motor test setup}
    \label{fig:servo} 
\end{figure}

To make the servo motor move the example-program for the motor class could be used \footnote{LOCATION TO MOTOR EXAMPLE PROGRAM}. Alternative the example-file readWrite form the dynamixel library could be used \footnote{LOCATION TO READWRITE EXAMPLE FILE}. 
For this to work there are some important parameters which should be known:
\begin{itemize}
     \item ID - Each motor has its own ID. This way its possible to control which motor that should get the command. The ID need to be change in the code to the ID of the motor. If the motor ID is unknown, you could use the \textit{pingAll()} function. This will search for IDs on the bus, and print out those that are active
     \item Port - Which port the USB2Dynamixel is connected to. This is used to set up the communication to the motors. USB devices under linux can be found under \textit{/dev/usb}. When testing we used deviceIndex = 0 (\textit{/dev/usb0}).
     \item Baudnumber - used to set the speed of the bus. This should always be one, which is 1Mbit/s, unless lower speeds are needed.
\end{itemize}
\vspace{\secspace}

By replacing these parameters in the example file the servos should move back and forth when enter is pressed.


\subsection{Driving}
The next step is to make four wheels cooperate!
This was implemented in the class Car. Four servomotors are mounted under the base of the robot. The two motors on the right turns the same way and the two on the left turns the other way. When creating a car object all you have to do is write the IDs of the wheels. The motors are initialized in endless turn mode when the car object is created. There was functions for setting speed and turning the car both when the car is driving and when it had stopped. Turning the car was accomplished by setting the two right wheels one way, and the two left wheels the other way when the car had stopped. When the car was moving only one side was set to a lower speed to make the car turn while moving.


\begin{figure}[H]
    \centering
    %sett inn figure her
    \caption{car overview}
    \label{fig:car} 
\end{figure}


In the example directory is an example program which demonstrates the car driving forward, backward and turning to left and right. The only thing needed are the IDs of the wheels. You can still use the pingAll() to check for IDs on the data-bus.

\subsection{Manipulator}

A manipulator was added to be able to pick up things and look around with the camera. The manipulator use three servomotors in the arm, and two in the gripper. The setup was like in the drawing below.

\begin{figure}[H]
    \centering
    %sett inn figure her
    \caption{manipulator overview}
    \label{fig:manipulator} 
\end{figure}

The class Manipulator implements the arm and the gripper.
\newline

\textbf{Arm}
\newline
The angles of the servomotors in the arm can be set directly by using the function setAngles($\theta_{1}$, $\theta_{2}$, $\theta_{3}$), or you can set the desired position in x, y and z, by using the function goToPosition(x,y,z). This function make use of inverse kinematic, with a geometric approach. 
The problem can be split up in to calculations, the calculation of $\theta_{2}$ and $\theta_{3}$, and the calculation of $\theta_{1}$. To calculate $\theta_{2}$ and $\theta_{3}$ we use the law of cosines.

\begin{figure}[H]
    \centering
    %sett inn figure her
    \caption{manipulator from the side}
    \label{fig:manipulator_side} 
\end{figure}

\begin{displaymath}
\begin{split}
-2d_{2}d_{3}\cos(\pi - \theta_{3}) = z_{c}^2 + r^2 - d_{2}^2 - d_{3}^2\\
r^2 = x_{c}^2 + y_{c}^2\\
\cos(\theta_{3}) = \frac{z_{c}^2 + x_{c}^2 + y_{c}^2 - d_{2}^2 - d_{3}^2}{2d_{2}d_{3}} = D\\
\sin(\theta_{3}) = \pm\sqrt{1-D^2}\\
\theta_{3} = \arctan(\frac{\pm\sqrt{1-D^2}}{D})
\end{split}
\end{displaymath}

\begin{figure}[H]
    \centering
    %sett inn figure her
    \caption{manipulator from the side}
    \label{fig:manipulator_side2} 
\end{figure}

\begin{displaymath}
\begin{split}
\theta_{2} = \frac{\pi}{2} - \alpha - \phi\\
\theta_{2} = \frac{\pi}{2} - \arctan(\frac{d_{3}\sin(\theta_{3})}{d_{2}+d_{3}\cos(\theta_{3})}) - \arctan(\frac{z_{c}}{r})
\end{split}
\end{displaymath}

Hence there are two solutions to the equation of $\theta_{3}$, one where $\theta_{3}$ is negative and one where $\theta_{3}$ is positive. The two solutions are called \textit{elbow up} and \textit{elbow down}. We chose the \textit{elbow up} solution because this would lead to less crashing with objects on the ground.
The length $d_{2}$ and $d_{3}$ had to be measured in mm and set in the code.

\begin{figure}[H]
    \centering
    %sett inn figure her
    \caption{manipulator from the top}
    \label{fig:manipulator_top} 
\end{figure}

The calculation of $\theta_{1}$ was simple.

$$\theta_{1} = \arctan(\frac{x_{c}}{y_{c}})$$

Since the servomotors only can move inside 0-300 degrees, the code had to include saturation limits on the angles, and will an error code if these limits are reached.
\newline

\textbf{Gripper}
\newline
The gripper used to servomotors on the end of the arm. The main problem with the gripper was to make it stop when it noticed it couldn't move further, and therefore not squeezing the object it had to grip. This was done by first setting the zero position where the two parts of the gripper could touch, and setting this as the first goal position. Then it had to read the position continuous until it noticed that it had reach the max number of consecutive reads. When this number was reached, the motors would stop ensuring that the object would not be destroyed.

\begin{figure}[H]
    \centering
    %sett inn figure her
    \caption{Gripper}
    \label{fig:gripper} 
\end{figure}

\textbf{Example}
\newline
The example file creates a manipulator with some given IDs and makes it move in y and z direction using inverse kinematic. It also demonstrates the gripper. The only thing needed are the IDs of the wheels. You can still use the pingAll() to check for IDs on the data-bus.

\subsection{Testing it all with a local interface}

mouse and keyboard input

x11/xming

\subsection{Sensor}

The use of the AX-S1 Integrated Sensor was realized by the Sensor class. Here are functions for calculating distance to objects using IR, functions for measuring light and functions for playing sound. Distance to objects are done by the sensor sending pulses with a IR-diode and measuring the strength of the IR returning. This way one can predict how close an object is (higher strength = close). 

The sensor can also play sound, either from its internal memory or by sending notes and how long it should play that note to it. In the datasheet is a table of notes and corresponding values. There are one function for each of these modes. The latter function takes in a array of notes and length of each note.

In the main program, thread --

\subsection{Exception handling}

failsafe mode

ping

threads

tried some wheels

\subsection{Future problems/challenges}
What happens if some of the wheels is disconnected? 
The answer for this question will not be covered in this report because of the timeframe, but it is an important issue that has to be resolved. 
The optimal solution is to find an algorithm that calculates the speed that the remaining wheels, so that the vehicle can obtain normal function even in failsafe mode. 
This will most likely depend on the surface under the vehicle. 


%%% Local Variables: 
%%% mode: latex
%%% TeX-master: "../report"
%%% End: 
