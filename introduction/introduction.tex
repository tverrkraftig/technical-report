\chapter{Introduction}
Introduction goes here...
% TODO: write an introduction

% TODO: translate the following:
Problemstillingen vår lyder som følger:
Lage en plattform som legger til rette for instrumentering og styring over internett. Plattformen skal være modulbasert slik at den er enkel å utvide til forskjellige  bruksområder. Det er også strenge krav til sikkerhet slik at uvedkommende ikke kan lese data eller sende styresignaler. 
Ulike redningstjenester opplever stadig situasjoner der det kan være vanskelig å bedømme hvor trygt det er for folk å gå inn ulike steder. Eksempler på dette er brennende hus der husets stand er ukjent, eller om det er uvisst om folk befinner seg der inne. Et annet eksempel er utforskning av grotter enten på land eller under vann. Skulle det skje noe med en dykker er redningsoppdraget vanskelig og tidkrevende (http://www.ranablad.no/nyheter/article7154581.ece). Til tross for at transmissionsmediumet er helt forskjellig under vann og i luft vil det være enkelt å utvide/forandre et slikt system.
I dag finnes det løsninger som funker bra i de forskjellige spesifikke scenariene, men det er ingen universell plattform som benyttes for alle løsningene. Ved hjelp av en slik plattform som er lett å utvide kan man få utviklet bedre og sikrere verktøy for slike situasjoner. Det er viktig at en slik plattform er stabil og driftsikker, derfor vil det være hensiktsmessig med kun en plattform som selv tilpasser seg til de sensorene/motorene som er koblet på. Slik vil vedlikehold av systemene holdes til et minimum. 
System skal i all hovedsak gå over internett. Dette kan være over vanlig WLAN eller 3G. Fordelen med et slikt system som går over internett er at man da kan bruke det uansett rekkevidde. En uavhengig enhet vil kunne styres fra hvor som helst i verden, når som helst. Slik kan også eksperter fra andre steder i verden hjelpe til med å vurdere en kritisk situasjon f.eks her i Norge. 

